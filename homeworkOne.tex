\documentclass[12px]{article}
\usepackage[utf8]{inputenc}
\usepackage[twoside, a4paper, left=3cm]{geometry}
\usepackage{ragged2e}
\usepackage{amsmath}



\begin{document}
\section*{3.2.1 KARAKTERİSTİK DENKLEMİN KÖKLERİ REEL VE FARKLI}\raggedright

\vspace*{10mm}
Bu alt bölümde, sabit katsayılı

\begin{equation*}
    ay^{''}+by^{'}+cy=0
\end{equation*}

diferansiyel denklemin $y=e^{rt}$ şeklinde çözümünün aranmasından ortaya çıkan 

\begin{equation*}
    ar^{2}+br+c=0
\end{equation*}
    
ikinci dereceden karakteristik denkleminin

\begin{equation*}
    \Delta=b^{2}-4ac > 0
\end{equation*}

olmak üzere

\begin{equation}
    r_{1,2}=\frac{-b\;\pm\;\sqrt{\Delta}}{2a}
\end{equation}

şeklinde reel ve farklı iki kökünün olması durumu incelenmektedir. Karakteristik bu iki kökü ile diferansiyel denklemin $y_{1}=e^{r_{1}^{t}}$ ve $y_{2}=e^{r_{2}^{t}}$ şeklinde farklı iki çözümü bulunur. Kısaca, karakteristik denklemin kökleri reel ve farklı olmak üzere diferansiyel denklemin genel çözümü 

\begin{equation*}
   y(t)=C_{1}y_{1}(t)+C_{2}y_{2}(t)
\end{equation*}

veya

\begin{quote}
    {\boldmath$y(t)=C_{1}e^{r_{1}^{t}}+C_{2}e^{r_{2}^{t}}$}
\end{quote}


formundadır.

\end{document}
