\documentclass{article}
\usepackage[utf8]{inputenc}
\usepackage{lipsum}
\usepackage{setspace}
\usepackage{xcolor}
\usepackage{rotating}
\usepackage{wrapfig}
\usepackage{colortbl}
\renewcommand{\tablename}{\textbf{Tablo}}

\begin{document}
\doublespacing
\begin{table}[h]
    \centering
    \begin{tabular}{|l|c|p{3cm}|}  % kac c,l,r varsa okadar sutun var 
        \hline % ust cizgi
        \rowcolor[rgb]{0.5,0.5,0.5}\multicolumn{3}{|c|}{\textbf{ANA BASLIK}}\\ [2ex] \cline{1-3}
        \rowcolor[gray]{0.3}\textbf{Baslik 1}   & \textbf{Baslik 2}  & \textbf{ Acıklama} \\ [2ex] \cline{1-3} 
        \rowcolor[gray]{0.9} \textcolor{dargray}{Hücre134234}        &  Hücre25656        & \dotfill \\ [2ex]
        \rowcolor[gray]{0.9} \textcolor{dargray}{Hücre354535}        &  Hücre4            & \dotfill \\ [2ex]
        \rowcolor[gray]{0.9} \textcolor{dargray}{Hücre5}             &  Hücre6767867      & \dotfill \\ [2ex] \cline{1-3}
    \end{tabular}
    \caption{Bu tablonun Aciklamasi}
    \label{firstTable}
\end{table}

\lipsum[1-4]
\begin{wraptable}{r}{8cm}
    \begin{tabular}{|l|l|p{2cm}|}  % kac c,l,r varsa okadar sutun var 
        \hline % ust cizgi
        \multicolumn{3}{|c|}{\cellcolor[gray]{0.2}\textbf{\textcolor{white}{ANA BASLIK}}}\\ [2ex] \cline{1-3} % altını cizme
        \cellcolor[gray]{0.5}{\textbf{\textcolor{white}{Baslik 1}}}   & \cellcolor[gray]{0.5}\textbf{\textcolor{white}{Baslik 2}}  & \cellcolor[gray]{0.5}\textbf{\textcolor{white}{Acıklama}}\\ [2ex] \cline{1-3}  % altını cizme
         Hücre134234        &  Hücre25656        & \dotfill \\ [2ex]
         Hücre354535        &  Hücre4            & \dotfill \\ [2ex]
         Hücre5             &  Hücre6767867      & \dotfill \\ [2ex] \cline{1-3}
    \end{tabular}
\end{wraptable}

\colorbox{black}{\textcolor{white}{Tablo \ref{firstTable}}} ve \colorbox{black}{\textcolor{white}{Tablo \ref{secondTable}}} elde edildi. \lipsum[4-5]  


\newpage
\begin{sideways}
        \begin{tabular}{|l|l|p{3cm}|}  % kac c,l,r varsa okadar sutun var 
        \hline % ust cizgi
        \multicolumn{3}{|c|}{\textbf{ANA BASLIK}}\\ [2ex] \cline{1-3} % altını cizme
        \textbf{Baslik 1}   & \textbf{Baslik 2}  & \textbf{ Acıklama}\\ [2ex] \cline{1-3}  % altını cizme
         Hücre134234        &  Hücre25656        & \dotfill \\ [2ex]
         Hücre354535        &  Hücre4            & \dotfill \\ [2ex]
         Hücre5             &  Hücre6767867      & \dotfill \\ [2ex] \cline{1-3}
    \end{tabular}
\end{sideways}

\lipsum[6-7]

\vspace{8cm}
\begin{rotate}{30}
        \begin{tabular}{|l|l|p{3cm}|}  % kac c,l,r varsa okadar sutun var 
        \hline % ust cizgi
        \multicolumn{3}{|c|}{\textbf{ANA BASLIK}}\\ [2ex] \cline{1-3} % altını cizme
        \textbf{Baslik 1}   & \textbf{Baslik 2}  & \textbf{ Acıklama}\\ [2ex] \cline{1-3}  % altını cizme
         Hücre134234        &  Hücre25656        & \dotfill \\ [2ex]
         Hücre354535        &  Hücre4            & \dotfill \\ [2ex]
         Hücre5             &  Hücre6767867      & \dotfill \\ [2ex] \cline{1-3}
    \end{tabular}
\end{rotate}


\end{document}

% r : saga yaslı sutun
% l : sola yaslı sutun
% c : ortaya hizalı sutun
% & : ayrı bir hucre olarak algilama
% \cline{1-3}: alt cizgi{sutun sayısı kadar}
% \hline : ust cizgi
% r|c|l : dusey cizgi
% \dotfill  : nokta ile hucreyı doldurma
% p{3cm} : tabloda istenilen uzunluk kadar yer ayırma(Aciklama gibi)
% [2ex] : satirin altinda space birakma
% \renewcommand{\tablename}{Tablo}  : \begin{table} da aciklama olarak Table yazdigindan onu otomatik olarak Tablo ile degistirdik
% \caption : table aciklamasi
% h veya !h(baskilamak) : Here yani yazılan yere tabloyu koyma eger h olmasa idi bir sonraki sayfaya tabloyu atardi.
% b : bottom yani sayfanın altına tabloyu koyar
% p : tabloyu tek basina ayri bir sayfaya at (buyuk tablolarda kullanılır)
% \vspace{10cm} : istenilen kadar bosluk birakma
% \usepackage{setspace} => \vspace{8cm} : tablo ustunden bosluk verme
% \usepackage{xcolor} => \colorbox{black}, \textcolor{white) : yazı color, background color renk verme
% \usepackage{rotating} => \begin{rotate}{30} : tabloyu istenilen derece dondurme
% \begin{sideways} : tabloyu yan cevirme
% \usepackage{wrapfig} => \begin{wraptable}{r}{8cm} : paragraf arasına tablo yerlestirme

