\documentclass[8pt, a4paper]{article}
\usepackage[left=1cm,right=1cm]{geometry}
\usepackage[utf8]{inputenc}
\usepackage{amsmath,nccmath}
\usepackage{diffcoeff}
\usepackage{fancyhdr}
\usepackage[T1]{fontenc}
\usepackage{paracol}

\renewcommand{\headrulewidth}{0pt}
\pagestyle{fancy} % the section draws their names and write top right
\lhead{E.Tosun}
\setlength\parskip{0pt}
\setlength\parindent{0cm}



\begin{document}
Subscripts preceded by a comma indicate partial differentiation with respect to material or spatial coordinates. 

In the absence of body forces, the equations of the motion in the reference state are 

\begin{fleqn}[\parindent]
    \begin{equation}
        \displaystyle T_{\Delta\beta,\Delta}+T_{3\beta,3}=0, \quad T_{\Delta3, \Delta}+T_{33,3}=\rhoü_3
    \end{equation}
\end{fleqn}

where $T_{Kk}$ are the components of the first Piola-Kirchhoff stress tensor field accompanying the deformation field $(1)$, a dot over $u_{3}$ indicates partial differentiation with respect to $\imath$, and $\rho=\rho(X_{\Delta})$ is the density of the layer. The boundary conditions can be written as

\begin{fleqn}[\parindent]
    \begin{equation}
        \displaystyle T_{2k}=0\quad on\quad X_2=h\quad and\quad u_3=0\quad on\quad X_2=0
    \end{equation}
\end{fleqn}

The components of the deformation gradient tensors $x_{k,K}$ and $X_{K,k}$ are as given in the list below

\begin{fleqn}[\parindent]
    \begin{equation}
        \displaystyle x_{\alpha,\Delta=\delta_{\alpha\Delta}},\quad x_{\alpha,3}=0,\quad x_{3,\Delta}=u_{3,\Delta},\quad x_{3,3}=1,
    \end{equation}
\end{fleqn}

\begin{fleqn}[\parindent]
    \begin{equation}
        \displaystyle X_{\Delta,\alpha}=\delta_{\Delta\alpha},\quad X_{\Delta,3}=0,\quad X_{3,\alpha}=-u_{3,\Delta}\delta_{\Delta\alpha},\quad X_{3,3}=1
    \end{equation}
\end{fleqn}

for the deformation field $(1)$ which is isochoric. $\displaystyle i.e.\ j=\det x_{k,K}={1.}$ In addition to using the components $(4)$ and $(5),$ if the relations $t_{kl}=j^{-1}x_{k,K}T_{Kl}$ and $T_{Kl}=jX_{K,k^{t},kl}$ are taken into consideration, then the components of the Cauchy stress tensor $t_{kl}$ and of the first Piola-Kirchhoff stress tensor $T_{Kl}$ can be written as

\begin{fleqn}[\parindent]
    \begin{equation*}
         \displaystyle t_{\alpha\beta}=\delta_{\alpha\Delta}T_{\Delta\beta},\quad t_{\alpha3}=\delta_{\alpha\Delta}T_{\Delta3},
    \end{equation*}
\end{fleqn}

\begin{fleqn}[\parindent]
    \begin{equation}
         \displaystyle t_{3\beta}=u_{3,\Delta}T_{\Delta\beta}+T_{3\beta},\quad t_{33}=u_{3,\Delta}T_{\Delta3}+T_{33},
    \end{equation}
\end{fleqn}

\begin{fleqn}[\parindent]
    \begin{equation*}
         \displaystyle T_{\Delta\beta}=\delta_{\Delta\alpha}t_{\alpha\beta},\quad T_{\Delta3}=\delta_{\Delta\alpha}t_{\alpha3},
    \end{equation*}
\end{fleqn}

\begin{fleqn}[\parindent]
    \begin{equation}
         \displaystyle T_{3\beta}=-u_{3,\Delta}\delta_{\Delta\alpha}t_{\alpha\beta}+t_{3\beta},\quad T_{33}=-u_{3,\Delta}\delta_{\Delta\alpha}t_{\alpha3}+t_{33},
    \end{equation}
\end{fleqn}

respectively. Therefore, the equations of the motion $(2)$ in terms of $t_{kl}$ are expressed as follows:

\begin{fleqn}[\parindent]
    \begin{equation*}
         \displaystyle (\delta_{\Delta\alpha}t_{\alpha\beta})_{,\Delta}+(-u_{3,\Delta}\delta_{\Delta\alpha}t_{\alpha\beta}+t_{3\beta})_{,3}=0,
    \end{equation*}
\end{fleqn}

\begin{fleqn}[\parindent]
    \begin{equation}
         \displaystyle (\delta_{\Delta\alpha}t_{\alpha3})_{,\Delta}+(-u_{3,\Delta}\delta_{\Delta\alpha}t_{\alpha3}+t_{33})_{,3}=\rhoü_{3}.
    \end{equation}
\end{fleqn}

If the layer consists of hyper-elastic materials, there exists a strain energy function $\sum$ which gives the mechanical properties of the constituent materials, and stress constituve equations can be given by

\begin{fleqn}[\parindent]
    \begin{equation}
         \displaystyle T_{Kk}=\frac{\partial\Sigma}{\partial x_{k,K}}.
    \end{equation}
\end{fleqn}



We consider that the constituve materials are isotropic and heterogeneous, so $\Sigma$ is the function of the principal invariants of the Finger deformation tensor $c^{-1}$ and $X_{\Delta}$, as 

\begin{fleqn}[\parindent]
    \begin{equation}
         \displaystyle I=trc^{-1},\quad 2II=(trc^{-1})^2-tr(c^{-2}),\quad III=detc^{-1}
    \end{equation}
\end{fleqn}

and calculated on the deformation field $(1)$ as

\begin{fleqn}
    \begin{equation}
        I=II=3+K^2,\quad III=1
    \end{equation}
\end{fleqn}

where $K^2=u_{3,\Delta}u_{3,\Delta}.$

Let us now assume that the heterogeneity is varied only with the depth and uniform in any direction parallel to the boundaries and consider generalized neo-Hookean materials. Hence, the strain energy function $\Sigma$ has a form

\begin{fleqn}
    \begin{equation}
        \Sigma=\Sigma(I,X_{2})
    \end{equation}
\end{fleqn}

Then the stress constituve equations are

\begin{fleqn}
    \begin{equation}
        \displaystyle t_{kl}=2\diff[1]\Sigma I (-\delta_{kl}+c_{kl}^{-1})
    \end{equation}
\end{fleqn}

where $c_{kl}^{-1}=x_{k,K^{x}l,K}$ are the components of Finger deformation tensor, found for the deformation field(1) as follows:

\begin{fleqn}
    \begin{equation}
        \displaystyle c_{\alpha\beta}^{-1}=\delta_{\alpha\Delta}\delta_{\beta\Delta},\quad c_{\alpha3}^{-1}=\delta_{\alpha\Delta}u_{3,\Delta},\quad c_{3\beta}^{-1}=u_{3,\Delta}\delta_{\beta\Delta},\quad c_{33}^{-1}=1+K^2.
    \end{equation}
\end{fleqn}




\end{document}
