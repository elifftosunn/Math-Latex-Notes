\documentclass[12px]{article}
\usepackage{lipsum}
\usepackage{setspace}
\usepackage[utf8]{inputenc}
\usepackage[left=2cm,twoside,a4paper]{geometry}
\usepackage{fancyhdr} % section isimlerini cizgi cekip ust saga yazıyor
\usepackage{indentfirst}

\setlength{\parskip}{0.5cm}  % 0.5-1 cm, 10pt gibi paragraflar arasında bosluk verme
%\setlength{\parindent}{1cm} % paragraf baslarının nekadar iceriden oldugunu soyluyoruz
\pagestyle{fancy} % section isimlerini cizgi cekip ust saga yazıyor
\lfoot{\LaTeX\ ile Metin Yazma} % sol alt kısma ozel olarak Latex yazma
%\rfoot{\TeX} % sag alt kısma ozel olarak Tex yazma
\rfoot{MAT218}
\lhead{\textsf{Elif TOSUN}}
\chead{\textsf{BEU}}
\rhead{\textsf{TEZ PROJECT}}

\begin{document}
\pagenumbering{roman}
\begin{center}
    \Huge{\textbf{GİRİŞ}}
\end{center}

\newpage
\begin{center}
    \Huge{\textbf{İÇİNDEKİLER}}
\end{center}


\newpage
\pagenumbering{arabic} % bu sayfa ve altı icin gecerli
\onehalfspacing % bu sayfa ve altı icin gecerli

\section*{ÖNSÖZ}
\begin{plain}
    \textsf{\lipsum[1-3]}\footnote{Bu bir açıklama notudur\ldots}
\end{plain}

\section*{AÇIKLAMA}
\begin{plain}
    \subsection{İLK AÇIKLAMA}
    \textsf{\lipsum[4-6]}\footnote{Bu da başka bir açıklama notudur\ldots}
    \subsection{İKİNCİ AÇIKLAMA}
    \textsf{\lipsum[7-9]}
\end{plain}

\section{KONU}
\begin{plain}
    \textsf{\lipsum[10-12]}
    
    \hfill\textbf{\textsf{Elif TOSUN}}
\end{plain}


\end{document}


% \begin{flushright} : yazıları saga yaslama
% \begin{flushleft} : yazıları sola yaslama
% \begin{plain} : yazıları iki yana yaslama
% \footnote : açıklama notu bırakma
% \ldots : cumle sonunda uc nokta bırakma
% \hfill : yazıyı yatay olarak en alta atar
% \vfill : yazıyı düsey olarak en alta atar
