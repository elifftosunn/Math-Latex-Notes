\documentclass{article}
\usepackage[utf8]{inputenc}
\usepackage{lipsum}
\usepackage{parallel,enumitem}
\usepackage{amsmath,nccmath}
\usepackage[left=1cm,right=1cm]{geometry}

\begin{document}


Bir $A$ kümesinin bir elemanı bir $B$ kümesine aitse $A$ kümesi $B$ kümesinin alt kümesidir denir ve $A \subset B$ 

$A=\{a,b,c\}$ olsun.  

$a_1,a_2$

$a^2+b^2=c^2$

$a_{n_1}$  $a_{23}$ $a^{45}$

$\alpha, \beta, \epsilon, \delta, \phi, \omega, \gamma, \sigma$

$\Delta, \Phi, \Omega, \Gamma, \Sigma$

Verilen her $\forall \epsilon>0$ sayısına karşılık $a<|x-x_0|<\delta$ iken $|f(x)-L|<\epsilon$ olacak biçimde $\exists \delta>0$ sayısı varsa $f$ fonksiyonunun $x_0$ noktasında limiti vardır ve $\lim_{x\to x_0} f(x)=L$ ile gösterilir.

\vspace{0.5cm}
$\sin x\quad cos x\quad tan x\quad cot x$

\vspace{0.5cm}
$\displaystyle \left\{\int_0^\pi \sin x\ dx\right\} \quad \left\{\int_0^x \cos x\ dx\right\}$

\vspace{0.5cm}
$\displaystyle \sum_{i=0}^n \frac{1}{i}=\left(1+\frac{1}{2}+\frac{1}{3}+\ldots+\frac{1}{n}\right)$

\vspace{0.5cm}
$x^{y}^{z}\quad {x^y}^z\quad x^{y^z}$  

\vspace{0.5cm}
$a_{n_k} \quad {a_n}_k \quad _{n^k} \quad a_n^k$

\vspace{0.5cm}
$\forall x \in X, \quad \exists y \leq \epsilon$

\vspace{0.5cm}
$\ln x \quad \log a \quad \exp(x)$

\vspace{1cm}

\begin{Parallel}[v]{0.5\textwidth}{0.5\textwidth}
\ParallelLText{$a^b=c$ }
\ParallelRText{ $\displaystyle \int_0^x \tan x=A$}
\end{Parallel}

%%%%%%%%%%%%%%%%%%%%%%%%%%%%%%%%%%%%%%%%%%%%%%%%%%%%%%%
\vspace{0.5cm} 
\begin{Parallel}[v]{0.5\textwidth}{0.5\textwidth}
\ParallelLText{$\displaystyle T_{\Delta\beta,\Delta}+T_{3\beta,3}=0, \quad T_{\Delta3, \Delta}+T_{33,3}=\rhoü_3$} 
\ParallelRText{$T_{2k}=0\quad on\quad X_2=h\quad and\quad u_3=0\quad on\quad X_2=0$}
\end{Parallel}

\vspace{0.5cm}
\begin{Parallel}[v]{0.5\textwidth}{0.5\textwidth}
\ParallelLText{$\displaystyle x_{\alpha,\Delta=\delta_{\alpha\Delta}},\quad x_{\alpha,3}=0,\quad x_{3,\Delta}=u_{3,\Delta},\quad x_{3,3}=1,$}
\ParallelRText{$X_{\Delta,\alpha}=\delta_{\Delta\alpha},\quad X_{\Delta,3}=0,\quad X_{3,\alpha}=-u_{3,\Delta}\delta_{\Delta\alpha},\quad X_{3,3}=1$}
\end{Parallel}

\vspace{0.5cm}
for the deformation field $(1)$ which is isochoric. $\displaystyle i.e.\ j=\det x_{k,K}={1.}$ In addition to using the components $(4)$ and $(5),$ if the relations $t_{kl}=j^{-1}x_{k,K}T_{Kl}$ and $T_{Kl}=jX_{K,k^{t},kl}$ are taken into consideration, then the components of the Cauchy stress tensor $t_{kl}$ and of the first Piola-Kirchhoff stress tensor $T_{Kl}$ can be written as

\vspace{0.5cm}
\begin{Parallel}[v]{0.5\textwidth}{0.5\textwidth}
\ParallelLText{$t_{\alpha\beta}=\delta_{\alpha\Delta}T_{\Delta\beta},\quad t_{\alpha3}=\delta_{\alpha\Delta}T_{\Delta3},$}
\ParallelRText{$t_{3\beta}=u_{3,\Delta}T_{\Delta\beta}+T_{3\beta},\quad t_{33}=u_{3,\Delta}T_{\Delta3}+T_{33},$}
\end{Parallel}

\vspace{0.5cm}
\begin{Parallel}[v]{0.5\textwidth}{0.5\textwidth}
\ParallelLText{$T_{\Delta\beta}=\delta_{\Delta\alpha}t_{\alpha\beta},\quad T_{\Delta3}=\delta_{\Delta\alpha}t_{\alpha3},$}
\ParallelRText{$T_{3\beta}=-u_{3,\Delta}\delta_{\Delta\alpha}t_{\alpha\beta}+t_{3\beta},\quad T_{33}=-u_{3,\Delta}\delta_{\Delta\alpha}t_{\alpha3}+t_{33},$}
\end{Parallel}

\vspace{0.5cm}
respectively. Therefore, the equations of the motion $(2)$ in terms of $t_{kl}$ are expressed as follows:

\vspace{0.5cm}
\begin{Parallel}[v]{0.5\textwidth}{0.5\textwidth}
\ParallelLText{$(\delta_{\Delta\alpha}t_{\alpha\beta})_{,\Delta}+(-u_{3,\Delta}\delta_{\Delta\alpha}t_{\alpha\beta}+t_{3\beta})_{,3}=0,$}
\ParallelRText{$(\delta_{\Delta\alpha}t_{\alpha3})_{,\Delta}+(-u_{3,\Delta}\delta_{\Delta\alpha}t_{\alpha3}+t_{33})_{,3}=\rhoü_{3}.$}
\end{Parallel}

\vspace{0.5cm}
If the layer consists of hyper-elastic materials, there exists a strain energy function $\sum$ which gives the mechanical properties of the constituent materials, and stress constituve equations can be given by

\vspace{0.5cm}
$\displaystyle T_{Kk}=\frac{\partial\Sigma}{\partial x_{k,K}}.$

\vspace{0.5cm}
We consider that the constituve materials are isotropic and heterogeneous, so $\Sigma$ is the function of the principal invariants of the Finger deformation tensor $c^{-1}$ and $X_{\Delta}$, as  

\vspace{0.5cm}
$I=trc^{-1},\quad 2II=(trc^{-1})^2-tr(c^{-2}),\quad III=detc^{-1}$

\vspace{0.5cm}
and calculated on the deformation field $(1)$ as

\begin{fleqn}
    \begin{equation}
        I=II=3+K^2,\quad III=1
    \end{equation}
\end{fleqn}


\begin{Parallel}
    \begin{fleqn}
        \begin{equation}
            \ParallelLText{\displaystyle I=II=3+K^2,\quad III=1}
        \end{equation}
    \end{fleqn}
\end{Parallel}





\end{document}



% $A$ : matematiksel ifade olarak yazma
% A \subset B  => Alt Kume
% ozel karakterlerde \ ifadesi kullanırız. \{a,b,c\} kume parantezi gibi
% \forall : her demek
% \quad : bosluk bırakma



