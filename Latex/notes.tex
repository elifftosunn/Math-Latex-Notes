\documentclass[12px]{article}
\usepackage{lipsum}
\usepackage{setspace} %paragraf araları icin 
\usepackage[left=4cm, right=2cm, top=2cm, twoside, a4paper]{geometry}

\setlength{\parskip}{12pt}
\setlength{\parindent}{1cm}
%\pagestyle{empty} % sayfa numaraları yok
%\pagestyle{headings} % sayfa numarası ust koseye gelecek
%\pagestyle{plain} % sayfa numarası alt ortada

\begin{document}

% Kapak Sayfası
\title{\Huge{\textbf{Title}}}
\maketitle
\large{Bugün \today{} ve yine \TeX perliğim üzerimde
\ldots}
\thispagestyle{empty}

% i. entry page
\newpage
\pagenumbering{roman}
\begin{center}
    \Huge{\textbf{GİRİŞ 1}}
\end{center}
\newpage

% ii. entry page
\newpage
\begin{center}
    \Huge{\textbf{GİRİŞ 2}}
\end{center}
\newpage

% i. contents page 
\newpage 
\pagenumbering{roman}
\begin{center}
    \huge{\textbf{İÇİNDEKİLER I}}
\end{center}
\newpage

% ii. contents page 
\newpage 
\begin{center}
    \huge{\textbf{İÇİNDEKİLER II}}
\end{center}
\newpage


% 1.page
\pagenumbering{arabic}
\onehalfspacing
\lipsum[1-2]
\singlespacing
\lipsum[3]

% 2.page
\doublespacing
\lipsum[4-8]

% 3-4.page
\onehalfspacing
\lipsum[11-15]
\end{document}



% \emph, \textit: italik yazı
% \textbf: koyu yazı
% \underline: altı çizili yazı
% \textsc: Baslık seklinde yazı
% \texttt: Daktilo seklinde yazı
% \textsf: Kıvrımsız Düz Yazı
% \textsl: Eğik Font'lu Yazı(sussuz)
% \begin quote: top-bottom paragraf boslugu
% \usepackage{lipsum}: hazır paragraf
% \hfill: paragrafın sonuna yaslamak icin kullanılır
% \- : taşan paragrafı kesme
% \'{e} : uzerinde ozel harf
% \^{a} : inceltilmis a

% \begin{itemize}: listeler olusturma
% \begin{enumerate}: Numaralı listeler olusturma
% \begin{center}: Yazıyı ortalama
% \begin{document}: belge bicimli yazma


% \usepackage{setspace} % paragraf araları icin library
% \doublespacing: çift satır aralığı
% \onehalfspacing: 1.5 satır aralığı
% \singlespacing: tek satır aralığı

% \setlength{\parskip}{1cm}: 0.5-1 cm, 10pt gibi paragraflar arasında bosluk verme
% \setlength{\parindent}{1cm}: paragraf baslarının nekadar iceriden oldugunu soyluyoruz (hangi paragraftan baslanırsa o paragraftan itibaren uyguluyor)
% \usepackage[left=2cm, right=2cm, top=3cm]{geometry}: sayfaya sagdan soldan bosluk verme
% \usepackage[left=4cm, right=2cm, top=2cm,twoside] twoside: arka arkaya çıktı alındığında metin hizalanmasında kullanılır.

% landscape: kagıdı yatay kullanma
% a4paper, a5paper, a3paper: kagıt boyutu
% \pagestyle{empty}: sayfa numaraları yok
% \pagestyle{headings}: sayfa numarası ust kosede
% \pagestyle{plain}: sayfa numarası alt ortada
% \thispagestyle{empty}: bulunulan sayfanın sayfa numarası yok
% ilk sayfada sayfa numarası yok daha sonraki sayfa 1'den baslıyor
% \newpage: yeni sayfaya gecis yapmak icin kullanılır
% \pagenumbering{roman}: i, ii, iii giris gibi sayfalarda sayfa numarası olarak yazma
% \pagenumbering{arabic}: 1,2,3 şeklinde sayfaları numaralandırma
% \pagenumbering{Alph}: a,b,c seklinde sayfaları numaralandırma (with big letter)

% \huge, \large, \normalsize ,\small: font size


\subsection{
\section{\textit{
\begin{enumerate}
\item \textbf{}
\begin{itemize}
\section{
\subsection{\textbf{\textit{
\begin{enumerate}
\item \[\item \(\)\textbf{}\]
\end{enumerate}
}}}
}
\end{itemize}
\end{enumerate}
}}
}

\begin{comment}
\thispagestyle{empty}
\singlespacing
\lipsum[3]

\doublespacing
\lipsum[4-8]\setcounter{page}{1}
\end{comment}




% 2.LESSON

\documentclass[12pt]{article}
\usepackage{lipsum}


\begin{document}
\begin{enumerate}
    \item İlk soruyu buraya yazıyorum. 
    \begin{enumerate}
        \item Bu a şıkkı olsun.
        \item Bu da b şıkkı olsun.
    \end{enumerate}
    \item İkinci soru da bunlar olacak. Sorudaki koşullar şunlardır:
    \begin{itemize}
        \item Soru anlamlı olacak.
        \item Öğrenci kolay çözecek.
    \end{itemize}
    \item Son soruyu buraya yazıyorum. 
    \begin{enumerate}
        \item[\textbf{a)}] Bu a şıkkı olsun.
        \item[\textbf{b)}] Bu da b şıkkı olsun.
        \item[\textbf{c)}] Bu da c şıkkı olsun.
    \end{enumerate}
\end{enumerate}
Yeni bir blok oluşturalım.
\begin{itemize}
    \item Birinci seçenek
    \begin{enumerate}
        \item Enumerate Seçenek 1
        \item Enumerate Seçenek 2
    \end{enumerate}
\end{itemize}

\begin{quote}
\begin{center}
    \textsc{Deneme Baslık Yazı}
    
    \texttt{Daktilo Seklinde Yazı}
    
    \textsf {Kıvrımsız Düz Yazı}
    
    \textsl {Eğik Font'lu Yazı}
    
    \textbf{Aşağıdaki seçenekler yapılabilir.}
\end{center}
\begin{enumerate}
    \item [\underline{\textbf{Seçenek 1.}}] İlk Seçenek
    \item [\underline{\textbf{Seçenek 2.}}] İkinci Seçenek
\end{enumerate}    
\end{quote}

\end{document}






Aşağıdaki seçenekler yapılabilir.
\begin{itemize}
    \item Bu yapılacaklar arasında ilk seçenektir.
    \begin{itemize}
        \item Birinci seçenekte de alt seçenekler var.
    \end{itemize}
    \item Bu yapılacaklar arasında ikinci seçenektir.
    \begin{itemize}
        \item Birden çok seçenek var. Bu birinci seçenektir.
        \begin{itemize}
            \item İkinci seçeneğin alt seçeneğinin alt seçeneği
        \end{itemize}
        \item İkinci seçenek bir başka alt seçenekte bulunur.
    \end{itemize}
    \item Bu yapılacaklar arasında sonuncu seçenektir.
\end{itemize}








% 1.LESSON


\lipsum[1]

\begin{quote} % begin{quote}: top-bottom paragraf boslugu
%Merhaba Dünya Bu bir giris metnidir.


\emph{Küçük prens çölü geçerken bir çiçeğe rastladı \underline{\emph{yalnızca}}. \emph{Üç taç\-yap\-rak\-lı} sıradan bir çiçekti.}

``Günaydın," dedi \underline{Küçük Prens.} 

``Günaydın," dedi çiçek.

``İnsanlar nerede?" diye \textbf{kibarca} \textit{sordu} Küçük \emph{Prens.}

``İnsanlar mı?" diye tekrarladı. ``Galiba altı, yedi \textbf{insan} var. Yıllar önce görmüştüm. Ama kim bilir şimdi neredeler? Rüzg\^{a}rla sürüklenmişlerdir. Kökleri yok. Yaşamları güç oluyor o yüzden."

\hfill --Küçük Prens, Antoine de Saint-Exepu\'{e}ry %paragrafın sonuna yaslamak icin kullanılır.
\end{quote}

\lipsum[2]




